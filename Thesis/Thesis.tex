\documentclass{article}
\usepackage[utf8]{inputenc}
\usepackage{indentfirst}
\usepackage{amsmath}

\title{Can a Shallow Ice Approximation Model be Used to Model the Water Output of Alpine Glaciers?}
\author{}
\date{}

\begin{document}

\maketitle

\section*{Abstract}
% (Insert abstract text here.)

\section{Introduction}
\subsection{Importance of glacial melting in mountain hydrology}
\subsection{Role of numerical modeling in understanding glacial runoff}
\subsection{Challenges in computational modeling of glaciers}

\section{Literature Review}
    As shown in Le Meur et al., 2004, there are significant differences in computational time between a SIA model and a Stokes model. When 
computing the free surface and associated velocity field, the SIA model took 1 minute of CPU time, whereas the Stokes model took 2 hours. 
This disparity grew even larger for 3D models. The authors show that there are some instances where SIA models do significantly worse than 
Stokes models, such as glaciers on steep slopes and glaciers in steep, narrow valleys because SIA models only approximate the Stokes 
equations. One of these approximations is to ignore horizontal stress gradients. This can cause a SIA model to deviate from a Stokes model 
significantly in predicted glacier flow and expansion. In one example, the resulting SIA model can have an upper free surface that is 
15--20\% greater than the Stokes model and velocities up to a factor of 2 greater (Le Meur et al., 2004). In the 2D model, the bed 
characteristics and slope become the limiting factor of the SIA model; Le Meur et al., 2004 note that the maximum velocity ratio of their 
SIA and Stokes models goes from 1.9 in a 3D model to 1.3 in a 2D model, which will differ depending on model configuration, but it tends to 
indicate that the horizontal stress gradients played a large part in this error. They found instances in which the SIA models performed well 
compared to Stokes models---particularly large flat glaciers with relatively free edges. One thing to note about this comparison study is 
that the authors are looking at the shape, area, and velocity profile of the glacier, whereas this study will focus on the water output 
(surface mass loss) of the glacier.

    There are several papers, such as Le Muer et al., 2003 and Kessler et al., 2006, that use an SIA model for alpine glaciers. The consensus 
from those papers is that SIA models only work well on alpine glaciers with a low aspect ratio, defined as the thickness-to-extent ratio in 
Le Muer et al., 2004. The glacier used by this study will have a low aspect ratio and therefore a SIA model should work well to model it.

    Additionally, there is precedent for using a SIA-Mass Balance model for modeling water runoff from glaciers (Naz et al., 2014). They used 
the SIA model to approximate the ice dynamics and a mass balance model to approximate the accumulation and ablation patterns on the glacier. 
As shown in their paper, the SIA model was able to accurately predict the glacier, and the coupled hydrological model was able to predict 
the stream flow accurately---only overestimating the July flow by an average of 13\% and underestimating the August and September flow by an 
average of 2\%.

\section{Thesis Statement}
% (Insert thesis statement here.)

\section{Methods}

\subsection{Study Site}
\textbf{South Cascade Glacier in the North Cascades of Washington}

\subsection{Physical Characteristics and Relevance}

\subsection{Availability of Hydrological and Glacial Data}

\subsection{Model Development}
\subsubsection{Model Overview}

\subsubsection{Ice Dynamics}
\paragraph{SIA Model} 
\paragraph{Assumptions}

\subsubsection{Snow and Rain Model}
The snow melt model uses precipitation and temperature data to melt and accumulate snow. This model uses 14 elevation bins and keeps track 
of the snow depth in each bin. The equation below is used to calculate the change in snow depth per timestep 
$$\text{snow depth} += 
\begin{cases} 
  p*\alpha & \text{if } T \leq 0,\\
  -\text{minimum}((s*T),\text{snow depth}) & \text{if } T > 0
\end{cases}$$
Where $p$ is the precipitation, $\alpha$ is the precipitation conversion factor, $s$ is the snow melt factor, and $T$ is 
the temperature. The snow melt is constrained so that there cannot be more melt than there is snow. The rain is simply modeled by $p*\alpha$ 
for positive temperatures

The total volume of snow is calculated by the equation
$$\text{snow melt volume}=(s*T)*(\text{area of bin}-\text{area of glacier})$$
This give us the total volume of snow melting off the glacier. The glacial melt is calculated elsewhere. The volume of rain is calculated by
$$\text{rain volume}=p*\alpha*(\text{area of bin})$$
This calculates the rain for the whole basin, assuming that any rain that falls off the glacier runs off immediately.
\subsubsection{Mass Balance Model}
The mass balance of the glacier is calculated using temperature and precipitation data from the Diablo Dam weather station at 272m. The 
temperature at the glacier is calculated by using a month-specific lapse rate. This month-specific lapse rate was empiraclly calculated using
data from the Diablo Dam weather station and the South Cascade Glacier weather station at 1830m from 2010-2018. The precipitation at the 
glacier is calculated by multiplying the precipitation at the Diablo Dam weather station by the precipitation conversion factor of 1.58 
(reference). The ablation of the glacier is calculated by using a combination of an ice melt factor and a snow melt factor. Above the ELA the 
ablation is calculated by the equation
$$\text{ablation}=T*\text{snow melt factor}$$
Below the ELA the ablation is calculated by the equation
$$\text{ablation}=T*(\text{snow melt factor}+((\text{ELA}-\text{elevation})/(\text{ELA}-\text{minimum(elevation)}))*$$
$$(\text{ice melt factor}-\text{snow melt factor}))$$
The result of this equation is the snow melt factor being used at the ELA and a linear increase in the melt factor until it hits the ice melt 
factor at the base of the glacier. 
The accumulation of the glacier is calculated using a similar linear equation that increases with time.
$$\text{accumulation}=p*\alpha*(\text{start accumulation}+((\text{year}-1984)/(2024-1984))*$$
$$(\text{start accumulation}-\text{end accumulation}))$$
This resuls in the accumulation increasing with time until it hits the end accumulation at 2024. 
\subsubsection{Glacial Melt Model}
The glacial melt model uses the mass balance of the glacier to calculate how much volume the glacier is losing. The volume of runoff from the 
glacier per timestep is calculated by the equation
$$\text{glacial melt volume}=(\text{mass balance}<0)*\text{area of glacier}$$
\paragraph{Data Used for Model}

\subsubsection{Model Calibration}

\subsubsection{Model Comparison}
\paragraph{Running OGGM Model for the Same Glacier}
\paragraph{Comparing Outputs of Both Models}
\paragraph{Validation Using Real-World Streamflow Data (Brunner et al., 2024)}

\section{Expected Results}
\subsection{Accuracy of SIA}
\subsection{Accuracy of OGGM}
\subsection{Comparison of Accuracy}

\section{Implications of Research}
\subsection{Importance of Simplified Ice Dynamics in Numerical Glacier Modeling}
\subsection{Applications}

\section{Discussion}
\subsection{What Worked}
\subsection{What Didn’t Work and Why}
\subsection{What Can Be Improved}

\section{Conclusion}
\subsection{Summary of Results}
\subsection{Conclusion of Model Accuracy}

\section{References}
% (Insert references here.)

\end{document}
